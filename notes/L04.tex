\documentclass[11pt]{article}
\usepackage{listings}
\usepackage{tikz}
\usepackage{enumerate}
\usepackage{url}
%\usepackage{algorithm2e}
\usetikzlibrary{arrows,automata,shapes}
\tikzstyle{block} = [rectangle, draw, fill=blue!20, 
    text width=5em, text centered, rounded corners, minimum height=2em]
\tikzstyle{bt} = [rectangle, draw, fill=blue!20, 
    text width=1em, text centered, rounded corners, minimum height=2em]

\lstset{ %
language=Java,
basicstyle=\ttfamily,commentstyle=\scriptsize\itshape,showstringspaces=false,breaklines=true,numbers=left}

\newtheorem{defn}{Definition}
\newtheorem{crit}{Criterion}

\newcommand{\handout}[5]{
  \noindent
  \begin{center}
  \framebox{
    \vbox{
      \hbox to 5.78in { {\bf Intro to Methods of Software Engineering } \hfill #2 }
      \vspace{4mm}
      \hbox to 5.78in { {\Large \hfill #5  \hfill} }
      \vspace{2mm}
      \hbox to 5.78in { {\em #3 \hfill #4} }
    }
  }
  \end{center}
  \vspace*{4mm}
}

\newcommand{\lecture}[4]{\handout{#1}{#2}{#3}{#4}{Lecture #1}}
\topmargin 0pt
\advance \topmargin by -\headheight
\advance \topmargin by -\headsep
\textheight 8.9in
\oddsidemargin 0pt
\evensidemargin \oddsidemargin
\marginparwidth 0.5in
\textwidth 6.5in

\parindent 0in
\parskip 1.5ex
%\renewcommand{\baselinestretch}{1.25}

\newcommand{\squishlist}{
 \begin{list}{$\bullet$}
  { \setlength{\itemsep}{0pt}
     \setlength{\parsep}{3pt}
     \setlength{\topsep}{3pt}
     \setlength{\partopsep}{0pt}
     \setlength{\leftmargin}{1.5em}
     \setlength{\labelwidth}{1em}
     \setlength{\labelsep}{0.5em} } }
\newcommand{\squishlisttwo}{
 \begin{list}{$\bullet$}
  { \setlength{\itemsep}{0pt}
     \setlength{\parsep}{0pt}
    \setlength{\topsep}{0pt}
    \setlength{\partopsep}{0pt}
    \setlength{\leftmargin}{2em}
    \setlength{\labelwidth}{1.5em}
    \setlength{\labelsep}{0.5em} } }
\newcommand{\squishend}{
  \end{list}  }

\begin{document}

\lecture{4 --- September 20, 2018}{Fall 2018}{Patrick Lam}{version 2}

[note: this was delivered October 23, 2018]

\section*{Software Engineering and Intellectual Property}

\noindent
Q: ``I'm just writing software, why do I care about Intellectual Property?''

\noindent
A: Turns out that software is important in the real world these days and \\
there are billion-dollar lawsuits about it, e.g.

\includegraphics[width=.4\textwidth]{images/L04-oracle_clr.png}
\begin{minipage}{2em} versus \vspace*{1.4em}
\end{minipage}
\includegraphics[width=.4\textwidth]{images/L04-g-header-2480.png}

\noindent
Q: ``What is Intellectual Property\footnote{Richard Stallman advocates not using that term at all: \url{https://www.gnu.org/philosophy/not-ipr.en.html}.}?''

\noindent
A: IP is:
\squishlist
\item a government-granted monopoly on certain actions;
\item an analogy to other types of property, e.g. real property, tangible property.
\squishend

\noindent
Q: What does ``government-granted monopoly'' mean?

\noindent
A: ``The Man'' is going to come and take you away if you do
not stop the offending behaviour.

\noindent
Q: Which actions are protected by IP law?

\noindent
A: depends on the kind of IP; see below.

\noindent
Q: Why does IP exist?

\noindent
A: US Constitution: ``to promote the Progress of Science and useful Arts, by securing for limited Times to Authors and Inventors the exclusive Right to their respective Writings and Discoveries.''

\noindent
If people can make money by writing books, we might see more people writing books.

\noindent
Q: Why is the analogy to other types of property misleading?

\noindent
A: per Thomas Jefferson: ``He who receives an idea from me, receives instruction himself without lessening mine; as he who lights his taper at mine, receives light without darkening me.'' Useful economics keyword: IP is ``non-rivalrous''.

\newpage
\noindent
Q: What kinds of IP are there?

\noindent
A: The most important kinds of IP for software engineers are:
\squishlist
\item copyrights;
\item patents;
\item trade secrets; and
\item trademarks.
\squishend

\paragraph{Side note.} At the University of Waterloo, inventors own the intellectual property
they create.

\section*{About Copyright}

\paragraph{Exclusive rights.} The copyright owner is allowed to prevent others from:
\squishlist
  \item producing and selling copies of the work;
  \item performing/displaying/transmitting the work; and
  \item creating derivative works.
\squishend

\paragraph{Applicability.} e.g. code, movies, literary works, maps. Not lists-of-facts,
e.g. phone books.

\paragraph{Ownership.} The author of the work, or for works-for-hire created in the
course of the author's employment, the employer. Can be sold.

\paragraph{Duration.} In Canada, creator's lifetime plus 50 years [to change to 70 under USMCA]. In the United States, creator's
lifetime plus 70 years.

\paragraph{Exceptions to copyright.} Fair dealing (Canada)/fair use (United States); public domain works; free software/Creative Commons materials.

\paragraph{Fair dealing/fair use.} Sets out (sort of complicated) exceptions to copyright protection, e.g. copying a short excerpt from a book for education is allowed. US fair use is more permissive than Canadian fair dealing.

\paragraph{Public domain.} Works on which copyright has expired/was waived can be freely used. (United States Government works are public domain, but not Canadian government works.)

\paragraph{Free software/Creative Commons.} Hack the copyright system to allow re-use. Work is still under copyright, but author grants permission to copy under certain conditions, e.g. must-attribute-author, or (GPL) use only in other GPL'd code.

\noindent (Some Open Educational Resources textbooks are coming out which help reduce textbook costs by being freely distributable.)

\section*{A note on plagiarism}
Plagiarism may seem related to copyright, but it differs in a number of ways.

Essentially, \emph{plagiarism is the use of materials for academic credit without permission.}

If you get caught plagiarising, you get a meeting with the instructor; typically for a first
case, the penalty is a 0 on the assignment and -5\% on the course score. We also document
the case so that we know about subsequent offenses.

\paragraph{Exercise.} Why might you be tempted to plagiarize? What should you do to avoid plagiarism?

\section*{Patents}

\paragraph{Applicability.} According to Industry Canada\footnote{\url{https://www.ic.gc.ca/eic/site/cipointernet-internetopic.nsf/eng/h_wr03652.html}}:

\noindent
``Patents cover new and useful inventions (product, composition, machine, process) or any new and useful improvement to an existing invention.''

\paragraph{Principle.} You share your invention with the world and get a limited-time monopoly on using your invention.

\paragraph{Software Engineering and Patents.} It's complicated, and varies country-by-country. Copyright protects particular implementations. Patents can (sometimes) protect a process that the computer is carrying out.

\paragraph{Patent trolls.} Some companies own patents but don't build things. Their business model
is to launch patent infringement lawsuits and settle cases for money.

\section*{Trade Secrets}
\paragraph{Example.} Coca-Cola recipe.

\paragraph{Principle.} A trade secret is information that is not disclosed to the world.
Trade secrets are protected against misappropriation (perhaps by a non-disclosure agreement)
but not against reverse engineering. Unlike patents, trade secrets do not expire.

\section*{Trademarks}
\paragraph{Examples.} Kodak, Starbucks.

\paragraph{Principle.} Protect consumers against confusingly-similar names.

Want to read more about IP (and why the term ``intellectual property'' is terrible?) Read this.

Samir Chopra, ``End Intellectual Property'', \emph{Aeon}. Accessed 20 December 2018. \url{https://aeon.co/essays/the-idea-of-intellectual-property-is-nonsensical-and-pernicious}

\end{document}

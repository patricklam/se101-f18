\documentclass[11pt]{article}
\usepackage[utf8]{inputenc}
\usepackage{textcomp}
\usepackage{listings}
\usepackage{tikz}
\usepackage{enumerate}
\usepackage{enumitem}
\PassOptionsToPackage{hyphens}{url}\usepackage{hyperref}
%\usepackage{algorithm2e}

\lstset{ %
  basicstyle=\ttfamily,commentstyle=\scriptsize\itshape,showstringspaces=false,breaklines=true,numbers=none}
\lstset{
     literate=%
         {á}{{\'a}}1
         {í}{{\'i}}1
         {é}{{\'e}}1
         {ý}{{\'y}}1
         {ú}{{\'u}}1
         {ó}{{\'o}}1
         {ě}{{\v{e}}}1
         {š}{{\v{s}}}1
         {č}{{\v{c}}}1
         {ř}{{\v{r}}}1
         {ž}{{\v{z}}}1
         {ď}{{\v{d}}}1
         {ť}{{\v{t}}}1
         {ň}{{\v{n}}}1                
         {ů}{{\r{u}}}1
         {Á}{{\'A}}1
         {Í}{{\'I}}1
         {É}{{\'E}}1
         {Ý}{{\'Y}}1
         {Ú}{{\'U}}1
         {Ó}{{\'O}}1
         {Ě}{{\v{E}}}1
         {Š}{{\v{S}}}1
         {Č}{{\v{C}}}1
         {Ř}{{\v{R}}}1
         {Ž}{{\v{Z}}}1
         {Ď}{{\v{D}}}1
         {Ť}{{\v{T}}}1
         {Ň}{{\v{N}}}1                
         {Ů}{{\r{U}}}1    
}

\newtheorem{defn}{Definition}
\newtheorem{crit}{Criterion}

\newcommand{\handout}[5]{
  \noindent
  \begin{center}
  \framebox{
    \vbox{
      \hbox to 5.78in { {\bf Intro to Methods of Software Engineering } \hfill #2 }
      \vspace{4mm}
      \hbox to 5.78in { {\Large \hfill #5  \hfill} }
      \vspace{2mm}
      \hbox to 5.78in { {\em #3 \hfill #4} }
    }
  }
  \end{center}
  \vspace*{4mm}
}

\newcommand{\lecture}[4]{\handout{#1}{#2}{#3}{#4}{Lecture #1}}
\topmargin 0pt
\advance \topmargin by -\headheight
\advance \topmargin by -\headsep
\textheight 8.9in
\oddsidemargin 0pt
\evensidemargin \oddsidemargin
\marginparwidth 0.5in
\textwidth 6.5in

\parindent 0in
\parskip 1.5ex
%\renewcommand{\baselinestretch}{1.25}

\begin{document}

\lecture{11 --- November 6, 2018}{Fall 2018}{Patrick Lam}{version 1}

\section*{Impostor syndrome}

\url{https://www.forbes.com/sites/mariaklawe/2018/10/08/lets-talk-about-impostor-syndrome/}

Maria Klawe is a Canadian computer scientist who is president of
Harvey Mudd, a small science and engineering undergraduate college in
California. (She also holds an honourary doctorate from the University
of Waterloo).

In her words:
\begin{quote}
  As new students take on the challenges of their first semester at
  college, many will experience self-doubt. They may feel like they
  don't belong in their new setting or wonder if they are as talented
  as the students around them or worry that soon everyone will find
  out that they are actually a failure. What they might not realize is
  that they are not alone—these feelings are incredibly common, and
  there's a name for them: impostor syndrome.
\end{quote}

\paragraph{What is Impostor Syndrome?}
A feeling of not deserving one's success, especially in the
presence of substantial success. Maybe you feel like you
got lucky, or maybe you think that you fooled others about your
abilities.

People report that impostor syndrome does not go away with increasing
success. It does seem to be more common for members of
underrepresented groups.

\paragraph{What to do about it?}
Here are some of Maria Klawe's tips:
\begin{enumerate}
\item Ask for help and accept it. (You are not weaker for asking for help; you can pay it forward by helping others also). Ask earlier rather than later.
\item Don't be defeated by feelings of failure. Failure is not something to fear.
\item Talk about your feelings.
\item Celebrate your successes. (Also a ritual that the Engineers without Borders encouraged at Fall Convocation last week).
\item Surround yourself with people who encourage you.
\item Try different things. ``Sometimes there's a doorway just a little to the side, and if I would just move over a bit I could open the door.''
\item Don't let your fears stop you from giving your best effort.
\end{enumerate}

She writes that it's in part a matter of perspective, and that failure is a powerful teacher, leading to perseverance.

\section*{Peers}
\begin{quote}
  Thank your classmates. They will be your friends for life.
\end{quote}
\hfill --- Pearl Sullivan at Spring Convocation, 2018.

Most of us can't do university on our own. As mentioned above, ask for help from your peers and from teachers
(TAs, instructors, etc).

You may feel that you're just receiving help from others. You'll have
a chance to help others later, either in your class or in lower-year
classes. Helping others can help you better consolidate your knowledge
too.

There's also not just help with school but also with emotional
issues. Be there to actively listen to your friends when they are having a
tough time; it happens to all of us.

As I emailed yesterday, I encourage you all to sign up for Hive, one
of the Capstone Design Projects from SE 2019. The goal is to facilitate
mentoring and networking relationships between students and we strongly
support this goal.

\paragraph{Peer Pressure.}
We all exist in society. I think it's a fundamental aspect of human nature
to exist in homogeneous societies; maybe it's a fact that we're more comfortable
with people who are like us, and we use peer pressure to actively enforce homogeneity.
[I am not a sociologist. Take SOC 101 if you want a professional opinion about this.]

I will admit that my inner voice likes to be judgey of people. ``Am I
better than X?  Worse than Y?'' However, I take active steps to not
act on these thoughts and tell myself that the thoughts are
inappropriate. There is a lot that I can't perceive, and writing
people off means that I'll miss out on a lot. I try to learn from everyone.

I think it's OK to say to myself ``I wouldn't do what person Z is doing''
but I think it's not OK to say ``Person Z shouldn't be doing that,''
especially to person Z. I think it's better to celebrate diversity and
how we are all different.

One caveat: this only applies as long as what people are doing is not
harmful to others or to the world. I think it's totally fine to be judgey
about someone who just drops their Kleenex on the ground.




\end{document}

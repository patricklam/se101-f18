\documentclass[11pt]{article}
\usepackage[utf8]{inputenc}
\usepackage{textcomp}
\usepackage{listings}
\usepackage{tikz}
\usepackage{enumerate}
\usepackage{enumitem}
\PassOptionsToPackage{hyphens}{url}\usepackage{hyperref}
%\usepackage{algorithm2e}

\lstset{ %
  basicstyle=\ttfamily,commentstyle=\scriptsize\itshape,showstringspaces=false,breaklines=true,numbers=none}
\lstset{
     literate=%
         {á}{{\'a}}1
         {í}{{\'i}}1
         {é}{{\'e}}1
         {ý}{{\'y}}1
         {ú}{{\'u}}1
         {ó}{{\'o}}1
         {ě}{{\v{e}}}1
         {š}{{\v{s}}}1
         {č}{{\v{c}}}1
         {ř}{{\v{r}}}1
         {ž}{{\v{z}}}1
         {ď}{{\v{d}}}1
         {ť}{{\v{t}}}1
         {ň}{{\v{n}}}1                
         {ů}{{\r{u}}}1
         {Á}{{\'A}}1
         {Í}{{\'I}}1
         {É}{{\'E}}1
         {Ý}{{\'Y}}1
         {Ú}{{\'U}}1
         {Ó}{{\'O}}1
         {Ě}{{\v{E}}}1
         {Š}{{\v{S}}}1
         {Č}{{\v{C}}}1
         {Ř}{{\v{R}}}1
         {Ž}{{\v{Z}}}1
         {Ď}{{\v{D}}}1
         {Ť}{{\v{T}}}1
         {Ň}{{\v{N}}}1                
         {Ů}{{\r{U}}}1    
}

\newtheorem{defn}{Definition}
\newtheorem{crit}{Criterion}

\newcommand{\handout}[5]{
  \noindent
  \begin{center}
  \framebox{
    \vbox{
      \hbox to 5.78in { {\bf Intro to Methods of Software Engineering } \hfill #2 }
      \vspace{4mm}
      \hbox to 5.78in { {\Large \hfill #5  \hfill} }
      \vspace{2mm}
      \hbox to 5.78in { {\em #3 \hfill #4} }
    }
  }
  \end{center}
  \vspace*{4mm}
}

\newcommand{\lecture}[4]{\handout{#1}{#2}{#3}{#4}{Lecture #1}}
\topmargin 0pt
\advance \topmargin by -\headheight
\advance \topmargin by -\headsep
\textheight 8.9in
\oddsidemargin 0pt
\evensidemargin \oddsidemargin
\marginparwidth 0.5in
\textwidth 6.5in

\parindent 0in
\parskip 1.5ex
%\renewcommand{\baselinestretch}{1.25}

\begin{document}

\lecture{9 --- October 23, 2018}{Fall 2018}{Patrick Lam}{version 1}

\section*{Volkswagen Emissions Scandal}

We are going to continue our ethics discussion with Volkswagen, which is 
highly germane to software engineers. 

\subsection*{Context}
There is some political will to reduce CO$_2$ emissions in response to
climate change.  Diesel produces maybe 10--20\% fewer CO$_2$ emissions
than gasoline per distance travelled, and more torque, but also more nitrous oxides
(NO$_\mathrm{x}$) and more fine particulate matter (also known as PM2.5, or
soot).

Historically, diesel has been popular in Europe (around 50\% of the
car market) but not in North America (around 3\% of the car market,
was growing until the scandal broke).  Yet people had been
observing higher soot emissions in cities, and the Paris mayor, Anne
Hidalgo, had called for diesel bans or buybacks, even prior to the
scandal. As of October 2018, diesel bans are coming in German cities:
Hamburg, Berlin, Stuttgart, Frankfurt.

\subsection*{The Issue}

Volkswagen's CEO had a stated objective of leading VW to be the biggest
car manufacturer in the world. However, to sell cars in the US, they had to meet
new, stricter NO$_\mathrm{x}$ limits. (NO$_\mathrm{x}$ aggravates asthma, among other
ill effects). Since 2008, they have been advertising ``clean diesels'' which allegedly meet
the new limits.

Catalytic converters reduce NO$_\mathrm{x}$ levels in gas engines, but
don't work for diesel engines. Instead, the usual solution is to use
``selective catalytic reduction'', which injects urea into the
combustion mixture. But the car then needs a urea tank and refills.

Or, in an illustration of a classic engineering tradeoff, the manufacturer
can sacrifice either power or fuel efficiency for cleaner emissions. Since it's 2016,
the tradeoff is controlled by software running in the engine control unit.

\subsection*{The Discovery}
Before the scandal broke, experts had been quietly suspicious about
the lack of an urea tank\footnote{Afterwards, Prof. Dudenhoffer, director of the Center for Automotive Research at the University of Duisburg-Essen, said ``They must have known that it's impossible, or else it's not possible they have degrees as engineers.''}; and customers had noticed soot on their
cars. 

In any case, it was odd that the US cars satisfied stronger standards
than those in Europe. The International Council on Clean
Transportation (ICCT), an engineering-heavy thinktank, wanted to know
why. It commissioned engineers at West Virginia University to
investigate.

VWU engineers Marc Besch and Arvind Thiruvengadam performed in-lab and
on-road testing of both VWs and BMWs. The in-lab tests, under
controlled conditions, achieved the advertised emissions targets. But once
they brought the VW cars on the road, they found NO$_\mathrm{x}$ emissions
that were $5\times$ to $35\times$ over targets. Not so for the BMWs. They presented
their work at the \emph{Real-World Emissions Workshop}, an academic conference,
in May 2014.

\subsection*{Regulators and Consequences}

The US Environmental Protection Agency started quietly investigating
soon afterwards.  Presumably after some period of discussions with
Volkswagen, they published a Notice of Violation in September 2015, at
which point the scandal made the news (see below). Although Volkswagen
owned up to the cheating (at some level) fairly early, it
continues to face massive consequences, including a steep decline in
sales, a change in CEO, a drop in stock value, and even raids of their
offices.

\section*{Engineering Analysis}

\subsection*{How did this work?}
\begin{itemize}[noitemsep]
\item it's all software-controlled;
\item when the software detects test conditions (no steering/fixed speeds/etc), it switches into so-called ``dyno calibration'' mode.
\item fewer emissions but also less torque and less efficiency.
\item must have been put in by software engineers working with the engine designers.
\end{itemize}

\subsection*{How did this happen?}
\begin{itemize}[noitemsep]
\item regulators set an objective testing scheme that was possible to hack (is being improved);
\item fundamentally, there is an engineering tradeoff: urea tank vs targets for torque and fuel efficiency.
\item Upper management set specifications which were not achievable; they attempted to blame rogue engineers; VW is an engineering-focussed organization, not credible.
\end{itemize}


\newpage
\section*{References}

I've only provided a superficial summary of the situation. There is a
lot of good long-form journalism on this topic. Good to read over Christmas.
Most links retrieved 21 November 2016.

\noindent
These are good general articles about the Volkswagen scandal.

\begin{itemize}
\item Russell Hotten, BBC. ``VW: The scandal explained.'' 10 December 2015. \url{http://www.bbc.com/news/business-34324772}
\item Damien McGuinness, BBC. ``VW emissions scandal hits 11m vehicles.'' 22 September 2015. \url{http://www.bbc.com/news/business-34325005}
\item Clifford Atiyeh, Car and Driver blog. ``Everything you need to know about the VW Diesel-Emissions Scandal.'' Updated 15 November 2016. 

\url{http://blog.caranddriver.com/everything-you-need-to-know-about-the-vw-diesel-emissions-scandal/}
\item Ezra Dyer, Popular Mechanics. ``This VW Diesel Scandal is Much Worse Than a Recall.'' 21 September 2015. \url{http://www.popularmechanics.com/cars/a17430/ezra-dyer-volkswagen-diesel-controversy/} (includes editorializing)
\item Daniel Beaulieu, autotrader.ca. ``Dieselgate from the Driver's Seat, Part 1.'' 22 June 2016. \url{http://www.autotrader.ca/newsfeatures/20160622/dieselgate-from-the-drivers-seat-part-1/#mJcbvZyFcUR6rimd.97} (Canadian first-person viewpoint)
\item Philip E. Ross, IEEE Spectrum. ``How Engineers at West Virginia University Caught VW Cheating.'' 22 Sep 2015. \url{http://spectrum.ieee.org/cars-that-think/transportation/advanced-cars/how-professors-caught-vw-cheating} (slightly more technical)
\end{itemize}

\noindent
About diesel in general:
\begin{itemize}
\item Richard Anderson, BBC. ``Diesel cars: What's all the fuss about?'' 16 September 2015. \url{http://www.bbc.com/news/business-34257424}
\item Damian Carrington, Gwyn Topham and Peter Walker, The Guardian. ``Revealed: nearly all new diesel cars exceed official pollution limits''. 23 April 2016. \url{https://www.theguardian.com/business/2016/apr/23/diesel-cars-pollution-limits-nox-emissions}
\item Taras Grescoe, NY Times. ``The Dirty Truth about 'Clean Diesel'.'' 2 January 2016. \url{http://www.nytimes.com/2016/01/03/opinion/sunday/the-dirty-truth-about-clean-diesel.html} (opinion piece about diesel)
\item Jill Petzinger, Quartz. ``Europe's intoxicating love affair with diesel is dying out.'' 22 January 2018. \url{https://qz.com/1183779/europes-intoxicating-love-affair-with-diesel-is-dying-out/} (2018 updates about diesel in Europe)
\item IFDC, via EurekAlert! ``Urea tanks on diesel trucks---that's the law in the United States starting in 2010''. 10 November 2008. \url{https://www.eurekalert.org/pub_releases/2008-11/i-uto111008.php}
\end{itemize}

\noindent
Earlier hints of defeat devices and diesel drawbacks:
\begin{itemize}
\item Arthur Neslen, The Guardian. ``European commission warned of car emissions test cheating, five years before VW scandal.'' 20 June 2016. \url{https://www.theguardian.com/environment/2016/jun/20/european-commission-warned-car-maker-suspected-cheating-five-years-vw-scandal}
\item John Voelcker, Green Car Reports. ``Are 'Clean Diesels' Actually Not Nearly As Clean As Claimed?'' 9 April 2015. \url{http://www.greencarreports.com/news/1097698_are-clean-diesels-actually-not-nearly-as-clean-as-claimed} (coverage before the headlines)
\item Brian Turner, driving.ca. ``To diesel or not to diesel? Pros and cons of diesel vs. gas.'' 23 March 2015. \url{http://driving.ca/volkswagen/golf/auto-news/news/to-diesel-or-not-to-diesel-that-is-the-question} (pre-scandal opinions re: diesel)
\item David Booth, driving.ca. ``Motor Mouth: Is diesel a sticky, ticking time bomb?'' 13 February 2015. \url{http://driving.ca/auto-news/entertainment/motor-mouth-stinky-time-bomb-why-diesel-is-at-a-crossroads} (notes Paris mayor is anti-diesel due to soot, pre-scandal)
\end{itemize}

\noindent
Long-form investigative reporting:
\begin{itemize}
\item William Boston, The Wall Street Journal. ``Volkswagen Emissions Investigation Zeros in on Two Engineers.'' 5 October 2015. \url{http://www.wsj.com/articles/vw-emissions-probe-zeroes-in-on-two-engineers-1444011602} (paywall)
\item Dune Lawrence, Bloomberg. ``How Could Volkswagen's Top Engineers Not Have Known?'' 21 October 2015. \url{http://www.bloomberg.com/news/articles/2015-10-21/how-could-volkswagen-s-top-engineers-not-have-known-} (contains an engineering analysis)
\item Geoffrey Smith and Roger Parloff. ``Hoaxwagen: How the massive diesel fraud incinerated VW's reputation---and will hobble the company for years to come.'' 7 March 2016. \url{http://fortune.com/inside-volkswagen-emissions-scandal/} Super detailed investigative reporting.
\end{itemize}

\noindent
Primary sources:
\begin{itemize}
\item Rachel Muncrief, The International Council on Clean Transportation. ``Defeat device testing in the EU: So far, not so good.'' 28 April 2016. \url{http://www.theicct.org/blogs/staff/defeat-device-testing-eu-so-far-not-so-good}
\item John German, The International Council on Clean Transportation. ``The emissions test defeat device problem in Europe is not about VW.'' 12 May 2016. \url{http://www.theicct.org/blogs/staff/emissions-test-defeat-device-problem-europe-not-about-vw}
\end{itemize}



\end{document}

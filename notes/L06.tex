\documentclass[11pt]{article}
\usepackage{listings}
\usepackage{tikz}
\usepackage{enumerate}
\usepackage{url}
\usepackage{enumitem}
%\usepackage{algorithm2e}
\usetikzlibrary{arrows,automata,shapes}
\tikzstyle{block} = [rectangle, draw, fill=blue!20, 
    text width=5em, text centered, rounded corners, minimum height=2em]
\tikzstyle{bt} = [rectangle, draw, fill=blue!20, 
    text width=1em, text centered, rounded corners, minimum height=2em]

\lstset{ %
language=C,
basicstyle=\ttfamily,commentstyle=\itshape,showstringspaces=false,breaklines=true,numbers=left}

\newtheorem{defn}{Definition}
\newtheorem{crit}{Criterion}

\newcommand{\handout}[5]{
  \noindent
  \begin{center}
  \framebox{
    \vbox{
      \hbox to 5.78in { {\bf Intro to Methods of Software Engineering } \hfill #2 }
      \vspace{4mm}
      \hbox to 5.78in { {\Large \hfill #5  \hfill} }
      \vspace{2mm}
      \hbox to 5.78in { {\em #3 \hfill #4} }
    }
  }
  \end{center}
  \vspace*{4mm}
}

\newcommand{\lecture}[4]{\handout{#1}{#2}{#3}{#4}{Lecture #1}}
\topmargin 0pt
\advance \topmargin by -\headheight
\advance \topmargin by -\headsep
\textheight 8.9in
\oddsidemargin 0pt
\evensidemargin \oddsidemargin
\marginparwidth 0.5in
\textwidth 6.5in

\parindent 0in
\parskip 1.5ex
%\renewcommand{\baselinestretch}{1.25}

\begin{document}

\lecture{6 --- October 11, 2018}{Fall 2018}{Patrick Lam}{version 1}

\section*{Feature Flags}

Also known as \emph{feature toggles}. Lots of information at
\url{https://martinfowler.com/articles/feature-toggles.html}.

Objectives for this lecture:
\begin{itemize}[noitemsep]
\item be able to apply feature flags to code that you've written.
\item understand need for feature flags in trunk-based development (the way that you'll use Git).
\end{itemize}

\subsection*{Situation}
Recall the Ideas Clinic exercise last week where you coded a spaceship simulation.
One of the subsystems was the \emph{navigation} system.
\begin{itemize}
\item First iteration: hard-code which warp gate to take.
\item Second iteration: implement Dijkstra's shortest path algorithm.
\end{itemize}

Let's think about progressing from the first iteration to the second iteration.
You had code for hard-coding decisions, which works in some universes.
But now you want to start working on Dijkstra's algorithm.

\paragraph{For Hero Programmers.}
Well, what if the navigation squad works only on their own computer?
What did you find to be the advantages and disadvantages of doing it that way?

\vspace*{7em}

% advantages
% well, it's easy and requires no understanding of software engineering

% disadvantages:
% aren't leveraging source control, e.g. can't go back in time
% hard to share; eventual issues with integration
% of course: limited to one computer!

\paragraph{Beyond Heroes.}
What you really wanted was a way to have both versions available for
other squads. They should be able to use the hard-coded version to see
whether their own code works in the first universe; and they should be
able to try out Dijkstra's algorithm as you develop it.

\subsection*{Three Options}
Let's say that the navigation squad wants to commit their work
to version control. Here are three things they could do:
\begin{enumerate}
\item comment out one solution at a time.
\item use revision control to switch back and forth between versions.
  % hard to make forward progress on trunk; doesn't really interact well with others' work.
\item use a boolean flag and conditional branch on the flag's value.
\end{enumerate}
Are these options good engineering? Why or why not---what are the
advantages and disadvantages? Think about whether they are fit for
purpose and fit for future.

\subsection*{Implementing Feature Flags}
Solution (3) above didn't talk about where you read the boolean flag
from. Here are three options. Let's say that we
started with this base code:

\begin{lstlisting}[language=C]
int navigate() {
  // hard-coded navigation implementation
  return 1;
}
\end{lstlisting}

\paragraph{Global constant/variable.}
This option is going to be enough for you until you start thinking
about writing more sophisticated programs, for instance in CS 247 in 2B.
It looks like this:
\begin{lstlisting}[language=C]
// feature flag
bool use_dijkstra = 1;

// move the old implementation here
int navigate_hardcoded() {
  return 17;
}

// new implementation goes here
int navigate_dijkstra() {
  // ...
}

int navigate() {
  // switch on the feature toggle
  if (use_dijkstra) {
    return navigate_dijkstra();
  } else {
    return navigate_hardcoded();
  }
}
\end{lstlisting}

\paragraph{Global flags array.} The next step up is to use an array.
This is more flexible.
\begin{lstlisting}[language=C]
#define NAV_DIJKSTRA 0
#define LAST_FLAG 1
bool feature_flags[LAST_FLAG];
  
// don't forget to call this early from main()!
void initialize_feature_flags() {
  feature_flags[NAV_DIJKSTRA] = 0;
}

// same navigate_hardcoded() and navigate_dijkstra() implementations

int navigate() {
  if (feature_flags[NAV_DIJKSTRA]) {
    return navigate_dijkstra();
  } else {
    return navigate_hardcoded();
  }
}
\end{lstlisting}

\paragraph{More indirection.} You can replace the statement
\begin{lstlisting}
  if (feature_flags[NAV_DIJKSTRA]) {
\end{lstlisting}
with a function call:
\begin{lstlisting}
  if (is_feature_flag_set(NAV_DIJKSTRA)) {
\end{lstlisting}
and then swap out the implementation of \verb+is_feature_flag_set+ with
something arbitrarily complicated, e.g. a read from a configuration
file or a database.

\subsection*{More advanced topics}
I'll briefly allude to two more advanced topics here.

We're going to talk about unit testing in the very near future. You can
write test cases that exercise both values of the feature flags: the old
test cases check the hard-coded navigation algorithm, while the new
test cases check the Dijkstra's algorithm.

There is a thing called A/B testing. Websites use it all the time.
The idea is to collect empirical data about, say, which ``Buy'' button
users are more likely to click on. Feature flags are one way to
implement A/B testing: you can have a flag which determines whether
the ``Buy'' button is red or green. [It'll be red, but it's good to run the experiment.]


\end{document}

\input{configuration-plam}

\title{Lecture 4 --- Software Engineering and\\ Intellectual Property}

\author{Patrick Lam}
\date{September 20, 2018}


\begin{document}

\begin{frame}
  \titlepage

 \end{frame}
 

\begin{frame}
\frametitle{Why Do I Care?}
\begin{center}
\Large
\vspace*{-3em}
Q: I'm just writing software. \\
Why do I care about Intellectual Property?

$\begin{array}{cc}\mbox{A:}\\~\\~\\
\end{array}$
\includegraphics[width=.4\textwidth]{images/L04-oracle_clr.png}
\begin{minipage}{2em} versus \vspace*{1.4em}
\end{minipage}
\includegraphics[width=.4\textwidth]{images/L04-g-header-2480.png}


\end{center}
\end{frame}

\begin{frame}
\Large
\begin{changemargin}{2cm}
Q: ``What is Intellectual Property\footnote{Richard Stallman advocates not using that term at all: \url{https://www.gnu.org/philosophy/not-ipr.en.html}.}?''

A: Intellectual Property is:\\[0em]
\begin{itemize}
\item a government-granted monopoly on certain actions;
\item an analogy to other types of property, e.g. real property, tangible property.
\end{itemize}
\end{changemargin}
\end{frame}

\begin{frame}
\Large
\begin{changemargin}{0.5cm}
Q: Why does IP exist?

\noindent
A: [per the US Constitution]
\begin{quote}
``to promote the Progress of Science and useful Arts, by securing for limited Times to Authors and Inventors the exclusive Right to their respective Writings and Discoveries.''
\end{quote}

\noindent
If people can make money by writing books, more people might write books.
\end{changemargin}
\end{frame}

\begin{frame}
\large
\begin{changemargin}{0.5cm}
\noindent
Q: Why is the analogy to other types of property misleading?

\noindent
A:
\only<2>{
 [per Thomas Jefferson] \begin{quote}
``He who receives an idea from me, receives insrtuction himself without lessening mine; as he who lights his taper at mine, receives light without darkening me.''
\end{quote}
Useful economics keyword: IP is ``non-rivalrous''.
}
\only<1>{
\begin{center}
\includegraphics[width=.4\textwidth]{images/L04-candle.jpg}
\end{center}
photo credit: Petar Milošević, Wikimedia Commons, CC-BY-SA 4.0
}

\end{changemargin}
\end{frame}


\begin{frame}
\Large
\begin{changemargin}{1.5cm}
\noindent
Q: What does ``government-granted monopoly'' mean?

\noindent
A: ``The Man'' will come and take you away if you do
not stop the offending behaviour.
\end{changemargin}
\end{frame}


\begin{frame}
\Large
\begin{changemargin}{1.5cm}
\noindent
Q: Which actions are regulated by IP law?

\noindent
A: It depends (on the kind of IP).
\end{changemargin}
\end{frame}

\begin{frame}
\Large
\begin{changemargin}{1.5cm}
\noindent
Q: What kinds of IP are there?

A: The most important kinds of IP for software engineers are:
\vspace*{-2em}
\begin{itemize}
\item copyrights;
\item patents;
\item trade secrets; and
\item trademarks.
\end{itemize}
\end{changemargin}
\end{frame}

\begin{frame}
\frametitle{Your IP}
\Large
\begin{changemargin}{1.5cm}
\vspace*{-2em}
\noindent
At the University of Waterloo, \\
inventors (i.e. you) own\\
the intellectual property they create.

(Though not while you're on co-op).
\end{changemargin}
\end{frame}

\part{About Copyright}
\begin{frame}
\partpage
\vspace*{-12em}\begin{center}\Huge \textcopyright\end{center}
\end{frame}


\begin{frame}
\frametitle{What does copyright do?}
\large
\begin{changemargin}{1cm}
\begin{center}
\includegraphics[width=.2\textwidth]{images/L04-do-not-enter.png}
\\
{\tiny (credit Fry1989, Wikimedia Commons, BY-SA 2.0)}
\end{center}
A copyright owner is allowed to prevent others from:
\vspace*{-1em}
\begin{itemize}
   \item producing and selling copies of the work;
   \item performing/displaying/transmitting the work; and
   \item creating derivative works.
\end{itemize}
\end{changemargin}

\end{frame}


\begin{frame}
\frametitle{What does copyright apply to?}
\Large
\begin{changemargin}{1cm}
Creative works.\\
Examples: code, movies, literary works, maps.

\vspace*{-2em}
Not lists-of-facts,
e.g. phone books.
\end{changemargin}
\end{frame}

\begin{frame}
\frametitle{Who first owns the copyright?}

\begin{changemargin}{1cm}
\Large
\vspace*{-1em}
The author of the work, or for works-for-hire created in the
course of the author's employment, the employer.

Can be sold.
\end{changemargin}
\end{frame}


\begin{frame}
\frametitle{How long does copyright last?}

\begin{changemargin}{1cm}
\Large
\vspace*{-1em}
In Canada, creator's lifetime plus 50 years. \\
In the US, creator's lifetime plus 70 years.
\end{changemargin}
\end{frame}

\begin{frame}
\frametitle{When can you copy?}

\begin{changemargin}{1cm}
\Large
Copyright protections are not absolute; exceptions:
\vspace*{-1em}
\begin{itemize}
\item fair dealing (Canada)/fair use (United States);
\item public domain works;
\item free software/Creative Commons materials.
\end{itemize}
\end{changemargin}
\end{frame}

\begin{frame}
\frametitle{Exceptions to copyright: fair~dealing/fair~use}

\Large
\begin{changemargin}{1cm}
Exceptions to copyright protection, \\
e.g. copying a short excerpt from a book for education is allowed.

(Also, ``classroom exception'': why I can show you videos).

US fair use is more permissive than \\
Canadian fair dealing.

\end{changemargin}
\end{frame}

\begin{frame}
\frametitle{Works not copyrighted: the Public Domain}

\Large
\begin{changemargin}{1cm}
Works on which copyright has expired/was waived can be freely used.

(United States Government works are public domain, but not Canadian government works.)

\end{changemargin}
\end{frame}

\begin{frame}
\frametitle{Copyleft: Free software/Creative Commons}

\Large
\begin{changemargin}{1cm}
Hack the copyright system to allow re-use.

Work is still under copyright, but author grants permission to copy under certain conditions, e.g. must-attribute-author, or (GPL) use only in other GPL'd code.

(Some Open Educational Resources textbooks are coming out which help reduce textbook costs by being freely distributable.)
\end{changemargin}
\end{frame}

\begin{frame}
\frametitle{Sidebar: Plagiarism}
\Large
\begin{changemargin}{1cm}
Plagiarism: not quite the same as copyright infringement.

\vspace*{1em}
\begin{quote}
\emph{Plagiarism is the use of materials for academic credit without permission.}
\end{quote}
\end{changemargin}
\end{frame}

\begin{frame}
\frametitle{Consequences of Plagiarism}
\Large
\begin{changemargin}{1cm}

What happens if caught:
\begin{itemize}
\item meeting with the instructor;
\item typical penalty = 0 on the assignment and -5\% on the course;
\item document case to punish subsequent offenses more severely.
\end{itemize}
\end{changemargin}
\end{frame}

\begin{frame}
\frametitle{Plagiarism Exercise: Some Questions}

\Large
\begin{changemargin}{1cm}
\vspace*{-2em}
Why might you be tempted to plagiarize?

What should you do to avoid plagiarism?
\end{changemargin}
\end{frame}

\part{About Patents}
\begin{frame}
\partpage
\end{frame}

\begin{frame}
\frametitle{What do patents apply to?}
\Large
\begin{changemargin}{1cm}
According to Industry Canada\footnote{\tiny \url{https://www.ic.gc.ca/eic/site/cipointernet-internetopic.nsf/eng/h_wr03652.html}}:

\vspace*{1em}
\begin{quote}
Patents cover new and useful inventions (product, composition, machine, process) or any new and useful improvement to an existing invention.
\end{quote}

That is: you share your invention with the world and get a limited-time monopoly on using your invention.
\end{changemargin}
\end{frame}

\begin{frame}
\frametitle{Software Engineering and Patents}
\Large
\begin{changemargin}{1cm}

It's complicated, and varies country-by-country.

Copyright protects particular implementations.

Patents can (sometimes) protect a process that the computer is carrying out.

\end{changemargin}
\end{frame}

\begin{frame}
\frametitle{Patent Trolls}
\Large
\begin{changemargin}{1cm}

\begin{center}
\includegraphics[width=.2\textwidth]{images/L04-troll.png}
\\
{\tiny (credit Ty Semaka for EFF, BY-SA 2.0; actually a copyright troll}
\end{center}
Some companies own patents \\
but don't build things.

Business model:\\
\hspace*{3em} launch patent infringement lawsuits\\
\hspace*{3em} and settle cases for money.

\end{changemargin}
\end{frame}

\part{About Trade Secrets}
\begin{frame}
\partpage
\end{frame}

\begin{frame}
\Large
\begin{changemargin}{1cm}

\begin{center}
\includegraphics[width=.4\textwidth]{images/L04-coca-cola.png}
\end{center}
A trade secret is information that is not disclosed to the world.

Protected from theft (perhaps by a non-disclosure agreement)
but not from reverse engineering.

Unlike patents, trade secrets do not expire.

\end{changemargin}
\end{frame}

\part{About Trademarks (™)}
\begin{frame}
\partpage
\end{frame}

\begin{frame}
\frametitle{What Trademarks Do}
\Large
\vspace*{-2em}
\begin{changemargin}{1cm}
Protect consumers against confusingly-similar names/identities.
\end{changemargin}
\end{frame}


\end{document}

